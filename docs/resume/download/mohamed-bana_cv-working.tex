% Options for packages loaded elsewhere
\PassOptionsToPackage{unicode}{hyperref}
\PassOptionsToPackage{hyphens}{url}
\PassOptionsToPackage{dvipsnames,svgnames,x11names}{xcolor}
%
\documentclass[
  a4paper,
]{article}
\usepackage{amsmath,amssymb}
\usepackage{lmodern}
\usepackage{setspace}
\usepackage{iftex}
\ifPDFTeX
  \usepackage[T1]{fontenc}
  \usepackage[utf8]{inputenc}
  \usepackage{textcomp} % provide euro and other symbols
\else % if luatex or xetex
  \usepackage{unicode-math}
  \defaultfontfeatures{Scale=MatchLowercase}
  \defaultfontfeatures[\rmfamily]{Ligatures=TeX,Scale=1}
\fi
% Use upquote if available, for straight quotes in verbatim environments
\IfFileExists{upquote.sty}{\usepackage{upquote}}{}
\IfFileExists{microtype.sty}{% use microtype if available
  \usepackage[]{microtype}
  \UseMicrotypeSet[protrusion]{basicmath} % disable protrusion for tt fonts
}{}
\makeatletter
\@ifundefined{KOMAClassName}{% if non-KOMA class
  \IfFileExists{parskip.sty}{%
    \usepackage{parskip}
  }{% else
    \setlength{\parindent}{0pt}
    \setlength{\parskip}{6pt plus 2pt minus 1pt}}
}{% if KOMA class
  \KOMAoptions{parskip=half}}
\makeatother
\usepackage{xcolor}
\usepackage{longtable,booktabs,array}
\usepackage{calc} % for calculating minipage widths
% Correct order of tables after \paragraph or \subparagraph
\usepackage{etoolbox}
\makeatletter
\patchcmd\longtable{\par}{\if@noskipsec\mbox{}\fi\par}{}{}
\makeatother
% Allow footnotes in longtable head/foot
\IfFileExists{footnotehyper.sty}{\usepackage{footnotehyper}}{\usepackage{footnote}}
\makesavenoteenv{longtable}
\setlength{\emergencystretch}{3em} % prevent overfull lines
\providecommand{\tightlist}{%
  \setlength{\itemsep}{0pt}\setlength{\parskip}{0pt}}
\setcounter{secnumdepth}{-\maxdimen} % remove section numbering
\usepackage{geometry}
\geometry{
    heightrounded,
    a4paper,
    inner=4mm,
    outer=4mm,
    top=4mm,
    bottom=14mm,
}

\usepackage{fontspec}
\defaultfontfeatures{
    Ligatures = { Required, Common, Contextual , Rare },
    Numbers = OldStyle,
    Mapping = tex-text,
}
\setmainfont{Iosevka Fixed SS05}[
    Scale=0.7,
]
\setsansfont{Iosevka Fixed SS05}[
    Scale=0.7,
]
\setmonofont{Iosevka Fixed SS05}[
    Scale=0.7,
]

\usepackage{multicol}


% \setmainfont[Scale=0.3,Mapping=tex-text,Ligatures={Common,Rare,TeX},Scale=MatchLowercase,Numbers=OldStyle]{Iosevka Regular}
% \setmainfont[Scale=MatchLowercase,Numbers=OldStyle,Ligatures={Common,Rare,Historic},Style=Swash]{Iosevka}

% \defaultfontfeatures{Mapping=tex-text,Ligatures={Common,Rare,TeX},Scale=MatchLowercase,Numbers=OldStyle}



% \setmainfont{/home/mbana/.fonts/truetype/iosevka-fonts/iosevka-regular.ttf}
% \setmainfont{/usr/share/fonts/iosevka-term-fonts/iosevka-term-regular.ttf}



% % \defaultfontfeatures{Mapping=tex-text,Numbers={OldStyle,Lowercase,Proportional},Ligatures={Common,Rare,TeX},Style=Historic}
% \defaultfontfeatures{Mapping=tex-text,Ligatures={Common,Rare,TeX},Scale=MatchLowercase,Numbers=OldStyle}
% \setmainfont[Mapping=tex-text,Ligatures={Common,Rare,TeX},Scale=MatchLowercase,Numbers=OldStyle]{Alegreya}
% \setsansfont[Mapping=tex-text,Ligatures={Common,Rare,TeX},Scale=MatchLowercase,Numbers=OldStyle]{Alegreya Sans}

% % \setmainfont[Ligatures = TeX, Scale = MatchLowercase, Numbers = OldStyle]{Alegreya}
% % \setsansfont[Numbers={OldStyle,Lowercase,Proportional},Ligatures={Common,Rare,TeX},Style=Historic]{Alegreya Sans}
% \setmonofont{DejaVuSansMono}
% % % \setmathfont{TeXGyreDejaVuMath}
% % \setmathfont{TeXGyreDejaVuMath-Regular}

% \usepackage{xunicode,xltxtra,url,parskip}
% \usepackage[usenames,dvipsnames]{xcolor}






% \documentclass[10pt,twocolumn]{revtex4}
% \usepackage[a4paper,margin=0.2cm]{geometry}

% %\usepackage{fontspec} % optional
% \usepackage{newtxtext,newtxmath}


% \usepackage{hyperref} % Required for adding links   and customizing them
% \definecolor{linkcolour}{RGB}{229,57,53} % Link color
% \hypersetup{colorlinks,breaklinks,urlcolor=linkcolour,linkcolor=linkcolour} % Set link colors throughout the document
\ifLuaTeX
  \usepackage{selnolig}  % disable illegal ligatures
\fi
\IfFileExists{bookmark.sty}{\usepackage{bookmark}}{\usepackage{hyperref}}
\IfFileExists{xurl.sty}{\usepackage{xurl}}{} % add URL line breaks if available
\urlstyle{same} % disable monospaced font for URLs
\hypersetup{
  pdftitle={Mohamed Bana's Curriculum Vitae},
  pdfauthor={I am only looking for a fully remote Golang - Senior Software Engineer - role},
  colorlinks=true,
  linkcolor={red},
  filecolor={Maroon},
  citecolor={Blue},
  urlcolor={Blue},
  pdfcreator={LaTeX via pandoc}}

\title{Mohamed Bana's Curriculum Vitae}
\author{I am \textbf{\emph{only}} looking for a fully remote Golang -
Senior Software Engineer - role}
\date{}

\begin{document}
\maketitle

% \usepackage{hyperref} % Required for adding links and customizing theme

\hypersetup{
  pdftitle={Mohamed Bana's Curriculum Vitae},
  pdfauthor={Mohamed Bana},
  pdfcreator={Mohamed Bana},
  pdfproducer={Mohamed Bana},
}

% \usepackage[pdftex,
%             pdfauthor={Your Name},
%             pdftitle={The Title},
%             pdfsubject={The Subject},
%             pdfkeywords={Some Keywords},
%             pdfproducer={Latex with hyperref, or other system},
%             pdfcreator={pdflatex, or other tool}]{hyperref}
\definecolor{linkcolour}{RGB}{229,57,53} % Link color
\hypersetup{colorlinks,breaklinks,urlcolor=linkcolour,linkcolor=linkcolour} % Set link colors throughout the document

\urlstyle{same}

% \renewcommand\UrlFont{\bfseries\itshape}
\renewcommand\UrlFont{\bfseries}

\setstretch{0.1}
\begin{longtable}[]{@{}
  >{\raggedright\arraybackslash}p{(\columnwidth - 2\tabcolsep) * \real{0.6343}}
  >{\raggedleft\arraybackslash}p{(\columnwidth - 2\tabcolsep) * \real{0.3657}}@{}}
\toprule()
\begin{minipage}[b]{\linewidth}\raggedright
\textbf{Contact}
\end{minipage} & \begin{minipage}[b]{\linewidth}\raggedleft
\textbf{Web}
\end{minipage} \\
\midrule()
\endhead
\href{tel:+44-7960-045-281}{+44-7960-045-281} &
\url{https://github.com/mbana} \\
\href{mailto:jobs@bana.io}{\nolinkurl{jobs@bana.io}} &
\url{https://linkedin.com/in/mbana} \\
\url{https://stackoverflow.com/users/241993/mohamed-bana} &
\url{https://bana.io/blog} \\
\bottomrule()
\end{longtable}

I am a highly skilled Software Engineer with 13 years of job experience
with a proven record. I love to build and maintain high-availability
rock-solid systems that support successful businesses. I am looking for
an interesting \textbf{remote-only} job, solo or in a small team of
professionals to share my knowledge and to learn from. Part-time
occupation is negotiable. Please read my cover letter at
\url{https://bana.io/resume/cover-letter}. To download my CV and/or
cover letter, please see \url{https://bana.io/resume/cv-download}.

\begin{multicols*}{2}

\hypertarget{work-experience}{%
\subsection{Work Experience}\label{work-experience}}

\hypertarget{senior-software-engineer-remote-cynergy-bank-london-united-kingdom---1303202313072023}{%
\subsubsection{\texorpdfstring{Senior Software Engineer (Remote),
\href{https://cynergybank.co.uk}{Cynergy Bank}, London, United Kingdom -
13/03/2023--13/07/2023}{Senior Software Engineer (Remote), Cynergy Bank, London, United Kingdom - 13/03/2023--13/07/2023}}\label{senior-software-engineer-remote-cynergy-bank-london-united-kingdom---1303202313072023}}

\textbf{Tech:} PostgreSQL, Postgres, Continuous Integration and
Continuous Delivery (CI/CD), OpenAPI Specification (OAS), Swagger API,
Google Cloud Platform (GCP), REST APIs, go, Linux, Golang, Go
(Programming Language), Docker, Git, Google Cloud Platform and
Kubernetes.

\hypertarget{senior-software-engineer-remote-kubeshop-greenwich-united-states---0105202220012023}{%
\subsubsection{\texorpdfstring{Senior Software Engineer (Remote),
\href{https://kubeshop.io/}{Kubeshop}, Greenwich, United States -
01/05/2022--20/01/2023}{Senior Software Engineer (Remote), Kubeshop, Greenwich, United States - 01/05/2022--20/01/2023}}\label{senior-software-engineer-remote-kubeshop-greenwich-united-states---0105202220012023}}

Worked on \href{https://docs.kusk.io/}{\texttt{kusk-gateway}}, a
OpenAPI-driven API Gateway for Kubernetes. Links:

\begin{itemize}
\tightlist
\item
  \url{https://kusk.kubeshop.io/}
\item
  \url{https://github.com/kubeshop/kusk-gateway}
\end{itemize}

Kusk is a Kubernetes API gateway powered by Envoy. The main difference
with other API Gateways is that Kusk has native support for OpenAPI.

Modern REST APIs are developed using OpenAPI specification that is then
used to generate API documentation, tests, server stubs and clients all
from the OpenAPI definition. Kusk enables the use of OpenAPI definitions
to configure the Ingress Controller of your Kubernetes clusters.

The commits I made against the repository is available to view at:
\url{https://github.com/kubeshop/kusk-gateway/commits?author=mbana}.

\textbf{Tech:} Golang, Kubernetes, Kubernetes Control Plane,
\href{https://github.com/envoyproxy/go-control-plane}{\texttt{go-control-plane}},
\href{https://www.envoyproxy.io/}{Envoy Proxy}, Docker, Docker Compose,
Minikube, Shell scripting/Bash, gRPC, Protocol Buffers, GitHub
Workflows, Linux.

\hypertarget{senior-software-engineer-remote-vitrifi-limited-london-uk---0111202103032022}{%
\subsubsection{\texorpdfstring{Senior Software Engineer (Remote),
\href{https://vitrifi.net}{ViTRiFi Limited}, London, UK -
01/11/2021--03/03/2022}{Senior Software Engineer (Remote), ViTRiFi Limited, London, UK - 01/11/2021--03/03/2022}}\label{senior-software-engineer-remote-vitrifi-limited-london-uk---0111202103032022}}

Cannot disclose details due to NDA.

\textbf{Tech:} Golang, Kubernetes, Docker, Docker Compose, AWS, Amazon
EKS, Grafana, Loki, Prometheus, Shell scripting/Bash, Visual Studio Code
Remote - Containers, gRPC, Protocol Buffers, Kafka, Redpanda Kafka,
GitLab.

\hypertarget{software-engineer-remote-paymentsense-limited-london-uk---0507202131102021}{%
\subsubsection{\texorpdfstring{Software Engineer (Remote),
\href{https://www.paymentsense.com}{Paymentsense Limited}, London, UK -
05/07/2021--31/10/2021}{Software Engineer (Remote), Paymentsense Limited, London, UK - 05/07/2021--31/10/2021}}\label{software-engineer-remote-paymentsense-limited-london-uk---0507202131102021}}

I worked with Golang on \texttt{Connect-E}
(\url{https://docs.connect.paymentsense.cloud/ConnectE/GettingStarted}).

\textbf{Tech:} Golang with modules, Docker, Docker Compose, TypeScript,
GCP, Google Cloud Datastore, Google Cloud Big Query, Google Cloud Pub /
Sub.

\hypertarget{software-engineer-remote-ibm-winchester-uk---1409202030042021}{%
\subsubsection{\texorpdfstring{Software Engineer (Remote),
\href{https://www.ibm.com/uk/en/}{IBM}, Winchester, UK -
14/09/2020--30/04/2021}{Software Engineer (Remote), IBM, Winchester, UK - 14/09/2020--30/04/2021}}\label{software-engineer-remote-ibm-winchester-uk---1409202030042021}}

I worked on IBM Cloud as Software Engineer on the IKS Cluster (IBM
Cloud™ Kubernetes Service) team as Cloud Software Engineer / Golang
Engineer. IKS is effectively something like AWS, GCP or Azure, see
\href{https://cloud.ibm.com/}{IBM Cloud}.

\begin{itemize}
\tightlist
\item
  I'm a member of the IBM Cloud™ Kubernetes Service team responsible for
  delivering IBM's Kubernetes managed container service. As a certified
  K8s provider, IBM Cloud Kubernetes Service provides intelligent
  scheduling, self-healing, horizontal scaling, service discovery and
  load balancing, automated rollouts and rollbacks, along with secret
  and configuration management and a fully managed image registry with
  integrated vulnerability scanning.
\item
  Working in an agile way and operating with a continuous delivery
  model.
\item
  Team/Squad consisted of around nine (9) people and we managed the
  complete life cycle of deliveries.
\end{itemize}

\textbf{Tech:} Go, Golang, Shell Scripting, Bash, Docker, Docker
Compose, Kubernetes, RedHat, \href{https://travis-ci.org/}{Travis CI},
\href{https://logdna.com/}{LogDNA},
\href{https://github.com/uber-go/zap}{go.uber.org/zap},
\href{https://etcd.io/}{etcd} or
\href{https://www.ibm.com/cloud/learn/etcd}{What is etcd? \textbar{}
IBM}.

\hypertarget{full-stack-software-engineer-open-banking-limited-london-uk---0805201801012020}{%
\subsubsection{\texorpdfstring{Full Stack Software Engineer,
\href{https://www.openbanking.org.uk}{Open Banking Limited}, London, UK
-
08/05/2018--01/01/2020}{Full Stack Software Engineer, Open Banking Limited, London, UK - 08/05/2018--01/01/2020}}\label{full-stack-software-engineer-open-banking-limited-london-uk---0805201801012020}}

Working as a full stack software engineer at Open Banking on a tool that
will validate a bank's implementation of the OpenBanking API spec, see:

\begin{itemize}
\tightlist
\item
  \url{https://github.com/OpenBankingUK/conformance-suite}
\item
  \url{https://bitbucket.org/openbankingteam/conformance-suite}
\item
  \url{https://hub.docker.com/u/openbanking/}
\end{itemize}

The commits I made against the repository is available to view at:
\url{https://github.com/OpenBankingUK/conformance-suite/commits?author=mbana}.

\textbf{Tech:} Go, Golang, Node.js, \href{https://vuejs.org/}{Vue.js},
\href{https://vuex.vuejs.org/}{Vuex}, Jest, Docker, Docker Compose,
Kubernetes, \href{https://openid.net/connect/}{OpenID Connect},
\href{https://jwt.io/}{JSON Web Token (JWT)},
\href{http://kompose.io/}{Kompose},
\href{https://circleci.com/}{CircleCI},
\href{https://swagger.io/}{Swagger}, WebSocket, Bitbucket Pipelines,
OpenAPI 3.0,
\url{https://bitbucket.org/openbankingteam/conformance-suite},
\url{https://hub.docker.com/u/openbanking/}.

\hypertarget{senior-software-engineer-90poe---ninety-percent-of-everything-limited-london-uk---0102201820042018}{%
\subsubsection{\texorpdfstring{Senior Software Engineer,
\href{https://www.90poe.io/}{90POE - Ninety Percent of Everything
Limited}, London, UK -
01/02/2018--20/04/2018}{Senior Software Engineer, 90POE - Ninety Percent of Everything Limited, London, UK - 01/02/2018--20/04/2018}}\label{senior-software-engineer-90poe---ninety-percent-of-everything-limited-london-uk---0102201820042018}}

I worked at startup specialising in software that runs on ship on two
projects that were heavily Go-based.

\hypertarget{platform-document-storage-service}{%
\paragraph{\texorpdfstring{\textbf{\texttt{platform-document-storage-service}}:}{platform-document-storage-service:}}\label{platform-document-storage-service}}

Document storage and retrieval to be used by others services, so it's a
core service. The core of the service was written in Go and exposed
using \href{https://grpc.io/}{gRPC} and http using
\href{https://github.com/gorilla/mux}{go gorilla/mux}. Both write and
retrieve supported arbitrarily large files which was achieved using gRPC
unidirectional streams. The underlying store was MongoDB's
\href{https://docs.mongodb.com/manual/core/gridfs/}{GridsFS} which I
interfaced with using the Go driver \href{https://gopkg.in/mgo.v2}{mgo}.

\hypertarget{auditing}{%
\paragraph{\texorpdfstring{\textbf{\texttt{auditing}}:}{auditing:}}\label{auditing}}

The service is structured very similar to the preceding in that the
underlying service is exposed using gRPC but the top-level interface is
done using \href{graphql.org}{GraphQL}. I wrote the GraphQL server in go
using \href{https://github.com/graph-gophers/graphql-go}{graphql-go}.
The underlying store is in Postgres and the library I used to interact
with it is \href{http://gorm.io/}{GORM}.

\textbf{Tech:} Go, Golang, \href{https://grpc.io/}{gRPC},
\href{https://github.com/gorilla/mux}{go gorilla/mux}, Protocol Buffers,
\href{http://gorm.io/}{GORM}, MongoDB,
\href{https://docs.mongodb.com/manual/core/gridfs/}{MongoDB GridsFS},
\href{https://gopkg.in/mgo.v2}{mgo}, \href{graphql.org}{GraphQL},
\href{https://github.com/graph-gophers/graphql-go}{graphql-go}, Docker,
Docker Compose, Kubernetes, NodeJS, Jest,
\href{https://concourse-ci.org/}{Concourse CI}, Postgres.

\hypertarget{full-stack-software-engineer-root-capital-llp-london-uk---0910201724122017}{%
\subsubsection{\texorpdfstring{Full Stack Software Engineer,
\href{https://www.rootcapital.co.uk}{Root Capital LLP}, London, UK -
09/10/2017--24/12/2017}{Full Stack Software Engineer, Root Capital LLP, London, UK - 09/10/2017--24/12/2017}}\label{full-stack-software-engineer-root-capital-llp-london-uk---0910201724122017}}

I worked as a full stack Node.js software engineer on the
\href{https://mindsforlife.com/}{Minds for Life} application, mainly on
the forum.

\hypertarget{frontend}{%
\paragraph{Frontend}\label{frontend}}

\begin{itemize}
\tightlist
\item
  \texttt{react}, \texttt{react-redux},
  \href{https://www.reactboilerplate.com/}{react-boilerplate}.
\item
  Single Page Application (SPA) targeting mobile platforms.
\item
  \texttt{ES56} using most of the latest ES56 features; \texttt{async},
  \texttt{await}, \texttt{classes} etc.
\item
  Serveless and hosted on \href{https://aws.amazon.com/s3/}{Amazon S3}
  as static assets, with \href{http://aws.amazon.com/cloudfront/}{Amazon
  CloudFront} as the CDN.
\end{itemize}

\hypertarget{backend}{%
\paragraph{Backend}\label{backend}}

\begin{itemize}
\tightlist
\item
  NodeJS server written in \texttt{ES6}, like the frontend.
\item
  \href{http://koajs.com}{Koa}.
\item
  MySQL as the datastore, using the \href{http://knexjs.org/}{Knex.js}
  library.
\item
  Packaged as a set of \texttt{docker} containers.
\end{itemize}

\hypertarget{ci-devops-and-infrastructure}{%
\paragraph{CI, Devops and
Infrastructure}\label{ci-devops-and-infrastructure}}

\begin{itemize}
\tightlist
\item
  Services were packaged as containers. Used \texttt{docker} and
  \texttt{docker-compose} to start them.
\item
  Builds managed by \href{https://semaphoreci.com}{Semaphore CI} and
  \href{http://www.wercker.com/}{Wercker}.
\end{itemize}

\textbf{Tech:} JavaScript, ES6, Node.js, react, react-redux,
\href{https://www.reactboilerplate.com/}{react-boilerplate}, Webpack,
\href{http://koajs.com}{Koa}, \href{http://knexjs.org/}{Knex.js}, Sequel
Pro, Docker, Docker Compose, Kubernetes,
\href{https://semaphoreci.com}{Semaphore CI},
\href{http://www.wercker.com/}{Wercker},
\href{https://aws.amazon.com/s3/}{Amazon S3},
\href{http://aws.amazon.com/cloudfront/}{Amazon CloudFront}

\hypertarget{full-stack-software-engineer-lloyds-banking-group-plc-london-uk---2003201722052017}{%
\subsubsection{\texorpdfstring{Full Stack Software Engineer,
\href{https://www.lloydsbankinggroup.com}{Lloyds Banking Group PLC},
London, UK -
20/03/2017--22/05/2017}{Full Stack Software Engineer, Lloyds Banking Group PLC, London, UK - 20/03/2017--22/05/2017}}\label{full-stack-software-engineer-lloyds-banking-group-plc-london-uk---2003201722052017}}

\hypertarget{nodejs---javascript}{%
\paragraph{NodeJS - JavaScript}\label{nodejs---javascript}}

\begin{itemize}
\tightlist
\item
  Loopback for server-side of the code.
\item
  ES5/6-based code base.
\end{itemize}

\hypertarget{monitoringdevopsmisc}{%
\paragraph{Monitoring/Devops/Misc}\label{monitoringdevopsmisc}}

\begin{itemize}
\tightlist
\item
  Splunk and sending logs via
  (\url{https://en.wikipedia.org/wiki/Syslog}) a LogDrain service
  available on Bluemix.
\item
  Gerrit for managing code. CI:
\item
  Jenkins: Configuring, managing and installing.
\item
  Jenkins 2: Same as previous plus writing pipeline scripts.
\end{itemize}

\textbf{Tech:} JavaScript, Node.js, ES5/6-based code base,
\href{https://loopback.io}{LoopBack} for server-side of the code,
\href{https://en.wikipedia.org/wiki/Syslog}{SysLog},
\href{http://www.splunk.com}{Splunk},
\href{https://www.jenkins.io}{Jenkins},
\href{https://www.jenkins.io/2.0}{Jenkins 2},
\href{https://www.gerritcodereview.com/}{Gerrit Code Review},
\href{https://en.wikipedia.org/wiki/Bluemix}{IBM Bluemix}

\hypertarget{senior-front-end-software-engineer-synthace-ltd.-london-uk---1304201604112016}{%
\subsubsection{\texorpdfstring{Senior Front-End Software Engineer,
\href{https://synthace.com}{Synthace Ltd.}, London, UK -
13/04/2016--04/11/2016}{Senior Front-End Software Engineer, Synthace Ltd., London, UK - 13/04/2016--04/11/2016}}\label{senior-front-end-software-engineer-synthace-ltd.-london-uk---1304201604112016}}

Did a fair amount of architectural UI work:

\begin{itemize}
\tightlist
\item
  JWT-based authentication: Implemented most, if not all, of the
  authentication related UI features. Polymer didn't have an
  authentication module as it's fairly new requiring me to re-implement
  this feature.
\item
  API interactions: I introduced Swagger JS and did the conversion from
  plain XHR to Promises, and ensured API was in-sync with the state of
  the authentication.
\item
  Updates via the Web Socket for notifications and async task updates:
  STOMP Over WebSocket.
\item
  Bootstrapped the testing using Web Component Tester.
\item
  Deciding on the build, test and hosting strategy, e.g., hosting our
  own CDN using Azure.
\item
  Performance: 1) pushed to have HTTP/2 enabled, and prototyped, on our
  custom server written in Go, 2) implemented lazy-loading of our Web
  Components which are included using HTML Imports, 3) Significantly
  improved UI build scripts; went from a somewhat un-deterministic build
  to one that almost always runs.
\item
  Introduced ES6 to the code-base, and moving to defining Polymer
  elements using ES6 classes.
\item
  Misc: libraries/utils to ease UI development.
\end{itemize}

We deploy Docker images to our Kubernetes cluster running in Azure
using:

\begin{itemize}
\tightlist
\item
  Docker: Fairly comfortable using this.
\item
  Kubernetes: I've done deployments of dev branches, so I understand the
  deployment model, navigating the Kubernetes dashboard and crude
  command line interactions, e.g., port-forwarding of the service the
  pod is running from the cluster to the local machine.
\end{itemize}

Added support our language called Antha to Monaco Editor, the editor
that powers Visual Studio Code.

Since this is a startup I have done a fair amount of work and I have
been given a fair share of responsibility, more so than any of my prior
roles.

\textbf{Tech:} JavaScript, ES6,
\href{https://www.polymer-project.org}{Polymer Project},
\href{https://www.webcomponents.org}{Web Components}, TypeScript,
VSCode, Azure, Docker, Kubernetes, Go,
\href{https://github.com/Polymer/web-component-tester}{Polymer Web
Component Tester}, Git, Swagger.

\hypertarget{software-engineer-ig-index-ltd.-london-uk---09-201418022016}{%
\subsubsection{\texorpdfstring{Software Engineer,
\href{https://www.ig.com/uk}{IG Index Ltd.} London, UK -
09-2014--18/02/2016}{Software Engineer, IG Index Ltd. London, UK - 09-2014--18/02/2016}}\label{software-engineer-ig-index-ltd.-london-uk---09-201418022016}}

\hypertarget{price-indicator-alerts}{%
\paragraph{Price \& Indicator Alerts}\label{price-indicator-alerts}}

My main responsibility was handling the UI aspect---setting,
configuring, triggering---of the various Alerts we support, from the
basic Price Alert to Technical Indicators such as RSI and Moving
Average. \textbf{Tech New:} Modern UI powered by ES2015 (Babel),
Ember.js, Handlebars and Less. The UI was then composed of isolated and
reusable Ember components. Runs on a NodeJS server, managed with npm and
bower, and version controlled using Git. Integration and unit tests
written in Mocha then run using Karma. \textbf{Tech Old:} Vanilla
JavaScript using an in-house framework when changing the previous UI.

\hypertarget{charts}{%
\paragraph{Charts}\label{charts}}

Assisted in the conversion of the Adobe Flex Real Time Charts to an
SVG-based version. \textbf{Tech:} d3 and tested like above.

\hypertarget{deeplinking}{%
\paragraph{Deeplinking}\label{deeplinking}}

A hashed action link, like typical deeplinks, that we send to our
Clients which then launches the mobile IG app, or directs them to the
app store for the device with the IG app pre-selected. Upon login the
action is carried out automatically. I did the bulk of the work with the
team lead overseeing it. \textbf{Tech:} Java 8, Spring and acceptance
tested using Cucumber. Redirecting and launching of the IG app was done
using vanilla JavaScript.

\hypertarget{software-engineer-itrs-international-trading-room-software-group-ltd.-london-uk---02-201009-2014}{%
\subsubsection{\texorpdfstring{Software Engineer,
\href{https://www.itrsgroup.com}{ITRS (International Trading Room
Software) Group Ltd.} London, UK -
02-2010--09-2014}{Software Engineer, ITRS (International Trading Room Software) Group Ltd. London, UK - 02-2010--09-2014}}\label{software-engineer-itrs-international-trading-room-software-group-ltd.-london-uk---02-201009-2014}}

\hypertarget{javascript-ui-software-engineer---012014092014}{%
\paragraph{JavaScript (UI) Software Engineer -
01/2014--09/2014}\label{javascript-ui-software-engineer---012014092014}}

UI for the next generation of the product which is built around, loosely
speaking, a real-time distributed analytics store. The aim is for the
old system to stream data to the new system so we can provide all the
great visualizations available in the HTML ecosystem, which was not
achievable in a reasonable amount of time in Swing.

The code is entirely modularized using RequireJS so we can test each
viewmodel without creating a view, we then test the entire UI
(end-to-end tests) using WebDriverJS.

Some Java/C coding required to write NodeJS bindings to interact with
the store. We evaluated several frameworks, Angular, Batman and
KnockoutJS, by writing prototype applications that connected to our
backend for a period of roughly 4-6 months before we chose to settle on
Knockout.js.

\textbf{Tech:} JavaScript, NodeJS, Node-WebKit, Durandal, KnockoutJS,
RequireJS, Git, Jasmine, Protractor (WebDriverJS), Jenkins, Bower,
HTML5, CSS, LoopBack.io.

\hypertarget{java-ui-software-engineer---072012122013}{%
\paragraph{Java (UI) Software Engineer -
07/2012--12/2013}\label{java-ui-software-engineer---072012122013}}

Worked with two software engineers and one QA member on a new Swing UI,
ACLite, that uses our new streaming-based API to access Geneos data, for
more info. see \url{https://resources.itrsgroup.com/OpenAccess}. We
access data from a fault-tolerant Cluster that is built on a set of
distributed nodes. The system we coded against is somewhat similar to
the Amazon distributed key/value store, DynamoDB, except with support
for streaming, so I am familiar with dealing with distributed systems.

\textbf{Tech:} Java, Maven, Swing, Eclipse, Jenkins, Git, Vagrant.

\hypertarget{c-software-engineer---022010062012}{%
\paragraph{C++ Software Engineer -
02/2010--06/2012}\label{c-software-engineer---022010062012}}

Spent one year with Run The Business (RTB) team, a team set-up to fix
critical bugs that Customers encounter. A very challenging role which
requires all-around product knowledge, good debugging skills and being
able to liaise with our Support staff in dealing with the Customers.
Prior to this I was one of three software engineers working in the
Transactions and Latency Monitoring team (part of the backend team)
doing core C++. We wrote and maintained the following plug-ins that are
part of the Geneos suite:

\begin{enumerate}
\def\labelenumi{\arabic{enumi}.}
\tightlist
\item
  \href{https://docs.itrsgroup.com/docs/geneos/5.1.0/Netprobe/trading_fix/fix.html}{Geneos
  FIX plug-in}: Monitors FIX (protocol) messages.
\item
  \href{https://docs.itrsgroup.com/docs/geneos/5.2.0/Netprobe/market_data_monitoring/feedadapter_ug.html}{Feed
  Latency Monitoring Plug-In}: Monitors feeds and calculates latency of
  instruments and fields across the monitored feeds.
\item
  \href{https://docs.itrsgroup.com/docs/geneos/5.2.0/Netprobe/latency_monitoring/message_tracker/fix_adapter.html}{Latency
  Monitoring - Message Tracker FIX adapter}: Tracks, generally, FIX
  messages across several checkpoints.
\item
  \href{https://docs.itrsgroup.com/docs/geneos/5.2.0/Netprobe/market_data_monitoring/mdm_ug.html}{Market
  Data Monitor}: See below. And bespoke plug-ins written for specific
  firms.
\end{enumerate}

We also maintained several non-finance specific plug-ins. I ported
another bespoke plug-in called Price Latency Monitor (provides latency
figures for bonds) to MS VC++ when I worked on this team. Projects:

\begin{itemize}
\tightlist
\item
  I converted the Windows version of the entire product suite from
  Visual Studio 2005 to 2010.
\item
  I wrote the Market Data Reliability plug-in. This plug-in connects to
  the Patsystem's Trading API to monitor commodity prices, using their C
  API, to determine if prices are `stale'.
\item
  I ported a significant part of our product to Solaris x86-64 (64-bit
  non-sparc architecture).
\end{itemize}

\textbf{Tech:} C++, STL, Boost, Visual Studio, Linux/Unix, GDB, DBX,
Make, Configure, XML, XPath, CPPUnit,
\href{https://en.wikipedia.org/wiki/Financial_Information_eXchange}{Financial
Information eXchange (FIX) protocol},
\href{https://docs.itrsgroup.com/docs/geneos/5.1.0/Netprobe/trading_fix/fix.html}{Geneos
FIX plug-in},
\href{https://docs.itrsgroup.com/docs/geneos/5.2.0/Netprobe/market_data_monitoring/feedadapter_ug.html}{Feed
Latency Monitoring Plug-In},
\href{https://docs.itrsgroup.com/docs/geneos/5.2.0/Netprobe/latency_monitoring/message_tracker/fix_adapter.html}{Latency
Monitoring - Message Tracker FIX adapter},
\href{https://docs.itrsgroup.com/docs/geneos/5.2.0/Netprobe/market_data_monitoring/mdm_ug.html}{Market
Data Monitor},
\href{https://docs.itrsgroup.com/docs/geneos/5.1.0/Netprobe/trading_pats/pats-status.html}{Geneos
PATS-STATUS Plug-In}, \href{www.patsystems.com}{Patsystem's Trading
API}.

\hypertarget{software-engineer-intern-then-tester-thomson-reuters.-london-uk---05-200911-2009}{%
\subsubsection{Software Engineer Intern, then Tester, Thomson Reuters.
London, UK -
05-2009--11-2009}\label{software-engineer-intern-then-tester-thomson-reuters.-london-uk---05-200911-2009}}

\hypertarget{c-software-engineer}{%
\paragraph{C\# Software Engineer}\label{c-software-engineer}}

One of four software engineers working on a search and navigation
interface to Global Product Search. Full life-cycle of development;
requirements engineering, analysis and design to implementation, testing
and deployment.

\textbf{Tech:} C\#, Silverlight 3.0, MS SQL Server 2005, LINQ, Web
Services (WCF), XML and Visual Studio 2008. I handled deployment using
CruiseControl.NET.

\hypertarget{tester}{%
\paragraph{Tester}\label{tester}}

User Acceptance Testing of the latest release of Thomson Reuter's 3000
Extra, then called UTAH, now called Eikon. UTAH combines the data from
Thomson and Reuters. My primary responsibilities were to validate the
end product against pre-defined requirements/workflows. 1. Worked on
Thomson Reuters project UTAH as part of a large team. 2. Tasks included
testing, observing, documenting software bugs, issues and errors before
final release of Utah. 3. Testing was done over multiple iterations.

\hypertarget{education}{%
\subsection{Education}\label{education}}

\begin{itemize}
\tightlist
\item
  \textbf{2005-2008:} BSc Computer Science (2.1), City, University of
  London.
\item
  \textbf{2008-2009:} MSc Software Systems Engineering (Attended),
  University College London, and Trading \& Financial Market Structure
  module, London Business School.
\end{itemize}

\hypertarget{additional-information}{%
\subsection{Additional Information}\label{additional-information}}

\hypertarget{misc}{%
\subsubsection{Misc}\label{misc}}

\begin{itemize}
\tightlist
\item
  \textbf{Passport/Nationality:} I am a British citizen with a British
  passport.
\item
  \textbf{Drivers Licence:} Full UK Driving Licence.
\end{itemize}

\hypertarget{languages}{%
\subsubsection{Languages}\label{languages}}

\begin{itemize}
\tightlist
\item
  \textbf{English and Swahili:} Native.
\item
  \textbf{Arabic:} Intermediate. I have lived in Marrakech, Morocco for
  almost two years. I have also lived in Cairo, Egypt and have travelled
  several times to the UAE.
\end{itemize}

\end{multicols*}

\end{document}
